\documentclass[10pt,a4paper]{article}
\usepackage[utf8]{inputenc}
\usepackage[dutch]{babel}
\usepackage{amsmath}
\usepackage{amsfonts}
\usepackage{amssymb}
\usepackage{graphicx}

\author{Pieter Van Keymeulen}
\title{NonkelScript}

\begin{document}
\maketitle
\newpage

\tableofcontents
\newpage

\section{Introductie}
Bedrijven zijn er in alle soorten en maten. Niet ieder bedrijf heeft dezelfde specialisatie en iedere onderneming verkoopt andere diensten naargelang hun aard. Neem nu een horecabedrijf in vergelijking met een supermarkt. Een restaurant heeft een variabel aanbod. Het menu kan van dag tot dag wisselen, evenals de prijzen die zelfs kunnen variëren van uur tot uur. Een supermarkt daarentegen heeft een vast aantal producten en de prijzen van deze producten variëren niet zo vaak.

Bij het maken van een ERP systeem is het belangrijk dat we kunnen inspelen op de noden van de klant. We kunnen een ERP systeem zodanig maken dat de klant dit door middel van scripts kan aanpassen. In ons geval spreken we van scripts die de klant in staat stellen fouten die gebeuren binnen het systeem op te vangen. Dit kan gaan van simpele dingen zoals een printer die zonder inkt komt te liggen, tot meer complexe omstandigheden zoals het blokkeren van de productie van goederen. In beide gevallen moet er een soort fout gegenereerd worden die afgevangen kan worden. De klant zal deze fout op zijn eigen manier willen afhandelen. Door middel van onze scripttaal is dat van nu af aan mogelijk.

Nonkelscript is een scriptingtaal met ingebouwde multithreading. Als de verdere ontwikkelingen het ons toelaten zal de Nonkelscript interpreter in de toekomst kunnen draaien als daemon. Deze daemon zal in staat zijn de verschillende subsystemen van het ERP systeem te ondervragen. Verder zal de daemon enkele scripts toegewezen krijgen die gekoppeld kunnen worden aan een bepaalde fout die het ERP systeem kan genereren.

\section{Programmeren binnen een ERP systeem}

%vertel hier over de api van NonkelErp

\section{Bison en flex}
Bison en Flex zijn twee programma's die tezamen gebruikt kunnen worden voor het maken van een compiler. Met Flex kan er via reguliere expressies een "tokenizer" gemaakt worden die gebruikt kan worden om broncode te scannen op bepaalde sleutelwoorden. De informatie de Flex op deze manier verzameld kan op zijn beurt weer gebruikt worden door Bison. Deze vraagt de gescande sleutelwoorden kunnen gebruikt worden voor het opstellen van regels geschreven in Bakkus-Naur vorm. Aan iedere regel kan er een actie gekoppeld worden. Dit is een stuk code. Meestal is deze code geschreven in C, maar aangezien onze compiler in Delphi geschreven zal worden zal deze code dan ook Delphi zijn. De parser die uit deze verzameling regels gegenereerd zal worden zal naar iedere regel zoeken telkens wanneer er één van die regels gevonden wordt de gekoppelde actie uitvoeren. Op deze manier kan er op een efficiënte wijze een interpeter gemaakt worden.

\subsection{Flex}
Zoals eerder gezegd werkt flex door middel van reguliere expressies. Deze expressies kunnen net zoals in Bison gekoppeld worden aan een actie. Deze actie wordt net zoals in de Bison parser uitgevoerd. In een standaard Flex scanner doen deze acties niet zoveel. Eigenlijk geven ze gewoon tokens terug die correspondeert met de gevonden reguliere expressie. 

\subsection{Bison}
Bison is ongeveer hetzelfde als flex. Met het verschil dat bison de informatie van die flex doorgeeft opvangt, verwerkt, en vergelijkt met een aantal regels. De regels worden geschreven in bakkus-naur vorm. Ook deze regels zijn gekoppeld aan een actie die uitgevoerd wordt wanneer de regel in de broncode gevonden wordt.

In ons geval gaan we voor iedere gevonden regel een node maken en deze node in onze syntaxboom stoppen. Meer hierover straks.

\section{Scripttaal}

\end{document}
