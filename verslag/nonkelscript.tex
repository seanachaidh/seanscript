\documentclass[10pt,a4paper]{article}
\usepackage[utf8]{inputenc}
\usepackage[dutch]{babel}
\usepackage{amsmath}
\usepackage{amsfonts}
\usepackage{amssymb}
\usepackage{graphicx}

\author{Pieter Van Keymeulen}
\title{NonkelScript}

\begin{document}
\maketitle
\newpage

\tableofcontents
\newpage

\section{Introductie}
Bedrijven zijn er in alle soorten en maten. Niet ieder bedrijf heeft dezelfde specialisatie en iedere onderneming verkoopt andere diensten naargelang hun aard. Neem nu een horecabedrijf in vergelijking met een supermarkt. Een restaurant heeft een variabel aanbod. Het menu kan van dag tot dag wisselen, evenals de prijzen die zelfs kunnen variëren van uur tot uur. Een supermarkt daarentegen heeft een vast aantal producten en de prijzen van deze producten variëren niet zo vaak.

Bij het maken van een ERP systeem is het belangrijk dat we kunnen inspelen op de noden van de klant. We kunnen een ERP systeem zodanig maken dat de klant dit door middel van scripts kan aanpassen. In ons geval spreken we van scripts die de klant in staat stellen fouten die gebeuren binnen het systeem op te vangen. Dit kan gaan van simpele dingen zoals een printer die zonder inkt komt te liggen, tot meer complexe omstandigheden zoals het blokkeren van de productie van goederen. In beide gevallen moet er een soort fout gegenereerd worden die afgevangen kan worden. De klant zal deze fout op zijn eigen manier willen afhandelen. Door middel van onze scripttaal is dat van nu af aan mogelijk.

Nonkelscript is een scriptingtaal met ingebouwde multithreading. Als de verdere ontwikkelingen het ons toelaten zal de Nonkelscript interpreter in de toekomst kunnen draaien als daemon. Deze daemon zal in staat zijn de verschillende subsystemen van het ERP systeem te ondervragen. Verder zal de daemon enkele scripts toegewezen krijgen die gekoppeld kunnen worden aan een bepaalde fout die het ERP systeem kan genereren.

\section{Programmeren binnen een ERP systeem}

%vertel hier over de api van NonkelErp

\section{Bison en flex}
%een kleine geschiedenisi over bison en flex

\subsection{Flex}
%Hoe flex in elkaar zit

\subsection{Bison}
%Hoe bison in elkaar zit

\section{Scripttaal}

\end{document}
