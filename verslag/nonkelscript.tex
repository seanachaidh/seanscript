\documentclass[10pt,a4paper]{article}
\usepackage[latin1]{inputenc}
\usepackage[dutch]{babel}
\usepackage{amsmath}
\usepackage{amsfonts}
\usepackage{amssymb}
\usepackage{graphicx}

\author{Pieter Van Keymeulen}
\title{NonkelScript}

\begin{document}
\maketitle
\newpage

\tableofcontents
\newpage

\section{Introductie}
Bedrijven zijn er in alle soorten en maten. Niet ieder bedrijf heeft dezelfde specialisatie en iedere onderneming verkoopt andere diensten naargelang hun aard. Neem nu een horecabedrijf in vergelijking met een supermarkt. Een restaurant heeft een variabel aanbod. Het menu kan van dag tot dag wisselen, evenals de prijzen die zelfs kunnen variëren van uur tot uur. Een supermarkt daarentegen heeft een vast aantal producten en de prijzen van deze producten variëren niet zo vaak.

Bij het maken van een ERP systeem is het belangrijk dat we kunnen inspelen op de noden van de klant. 

\section{Programmeren binnen een ERP systeem}

\section{Bison en flex}

\subsection{Flex}

\subsection{Bison}

\section{Scriptingtalen}

\end{document}
